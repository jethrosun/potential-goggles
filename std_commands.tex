%%%%%%%%%%%%%%%%%%%%%%%%%%%%%%%%%%%%%%%%%%%%%%%%%%%%%%%%%%%%%%%%%%%%%%

%-- DISCLAIMER:
%-- This is open sourced LaTeX utils that I adapted and changed over time, you
%-- are free to change or use it.     

%%%%%%%%%%%%%%%%%%%%%%%%%%%%%%%%%%%%%%%%%%%%%%%%%%%%%%%%%%%%%%%%%%%%%%

%-- place any standard commands/environments here to get included in
%-- documents.  When you include this file, you should do it before
%-- the \begin{document} tag.


%-- NECESSARY ENVIRONMENTS OR PACKAGES (paste in to main.tex if necessay):



%%%%%%%%%%%%%%%%%%%%%%%%%%%%%%%%%%%%%%%%%%%%%%%%%%%%%%%%%%%%%%%%%%%%%%

%-- CHANGES:
%-- 07/31/01 -jstrunk- Added command to set the paper margins.
%-- 09/25/17 -jethros- Added some changes around todo.
%-- 11/23/18 -jethros- Organized things so that it will be easier to use.


%-- Provides fixed width font for commands and code snips.
% ???
%-- The bit that need to paper.tex right away:
% =========================================================

%-- Necessary packages... To enable std_commands
\usepackage[usenames,dvipsnames,svgnames,table]{xcolor}
\usepackage{array}
\usepackage{ifthen}
\usepackage{xspace}
\usepackage{enumitem}
\usepackage{cite}
\usepackage[colorlinks=true,citecolor=Bittersweet,linkcolor=black]{hyperref}% violetred

%-- To use booktabs...
\usepackage{booktabs}
\usepackage{rotating} % Rotate tables
\usepackage{tabularx}
\usepackage{multirow}
\usepackage{chngpage}
\usepackage{adjustbox}
\usepackage[english]{babel}
\usepackage{rotating} % Rotate tables
\newcolumntype{P}[1]{>{\centering\arraybackslash}p{#1}}
\usepackage{graphicx}
\usepackage[font={small,it}]{caption}
\usepackage{subcaption}
\usepackage{lscape}
\usepackage{pifont}

%-- 0 means no comments, and 1 o/w
\newcommand{\showcomments}{0}

%-- enable todo
%\providecommand{\todo}[1]{\textcolor{red}{TODO: #1}\PackageWarning{TODO:}{#1!}}
%\renewcommand{\todo}[1]{}   % enable it to supress TODO

% ---------------------------------------------------

%-- Save space...
\usepackage[subtle]{savetrees}
%\usepackage[moderate]{savetrees}
%\usepackage[extreme]{savetrees}

%-- Links... change ref link color
\definecolor{blue1}{rgb}{0.01, 0.28, 1.0}
\DeclareCaptionFont{blue}{\color{blue1}}
\captionsetup[figure]{labelfont=blue}

%-- Booktabs...
\newcommand{\ra}[1]{\renewcommand{\arraystretch}{#1}}
\newcolumntype{L}[1]{>{\raggedright\arraybackslash}p{#1}}
\newcommand{\head}[1]{\textbf{#1}}
\setlength{\extrarowheight}{4pt}
\newcommand{\normal}[1]{\multicolumn{1}{l}{#1}}

%%%%%%%%%%%%%%%%%%%%%%%%%%%%%%%%%%%%%%%%%%%%%%%%%%%%%%%%%%%%%%%%%%%%

%-- Macros... examples...
\newcommand{\netex}{FlexNet\xspace}
\hyphenation{FlexNet}

%-- ????
\renewcommand{\sectionautorefname}{\S}
\renewcommand{\subsectionautorefname}{\S}

%-- Code... Not sure this is useful...
\newcommand{\code}[1]{\texttt{\textbf{#1}}}

%-- Terms...  Use this to introduce a term in the paper.
\newcommand{\term}[1]{\emph{#1}}

%-- Provides stylization for e-mail addresses
%\newcommand{\email}[1]{\emph{(#1)}}

%-- Starts a minor section (puts the title inline w/ the text.
\newcommand{\minorsection}[1]{\textbf{#1}:}

%-- Jiri caption
\newcommand{\minicaption}[2]{\caption[#1]{\textbf{#1.} \textnormal{#2}}}

%-- Units on numbers: 4KB -> \units{4}{KB}
\newcommand{\units}[2]{#1~#2}

%-- Commands...  i.e. WRITE commands.
\newcommand{\command}[1]{{\sc \MakeLowercase{#1}}}

%-- For notes about things that need to be fixed.
\newcommand{\fix}[1]{\marginpar{\LARGE\ensuremath{\bullet}}
\MakeUppercase{\textbf{[#1]}}}
%-- For adding inline notes to a draft preceded by your initials
%-- E.g., \fixnote{JJW}{What the heck is a foobar?}
\newcommand{\fixnote}[2]{\marginpar{\LARGE\ensuremath{\bullet}}
{\textbf{[#1:} \textit{#2\,}\textbf{]}}}

%-- Setting margins: \setmargins{left}{right}{top}{bottom}
\newcommand{\setmargins}[4]{
  % Calculations of top & bottom margins
  \setlength\topmargin{#3}
  \addtolength\topmargin{-.5in}  %-- seems like this should be 1, but .5
  %-- balances the text top to bottom
  \addtolength\topmargin{-\headheight}
  \addtolength\topmargin{-\headsep}
  \setlength\textheight{\paperheight}
  \addtolength\textheight{-#3}
  \addtolength\textheight{-#4}

  % Calculations of left & right margins
  \setlength\oddsidemargin{#1}
  \addtolength\oddsidemargin{-1in}
  \setlength\evensidemargin{\oddsidemargin}
  \setlength\textwidth{\paperwidth}
  \addtolength\textwidth{-#1}
  \addtolength\textwidth{-#2}
}

%-- For the tabularx environment... Using L, C, R as the column type
%-- will left, center, or right justify the text.
\newcolumntype{L}{X}
\newcolumntype{C}{>{\centering\arraybackslash}X}
\newcolumntype{R}{>{\raggedleft\arraybackslash}X}

%-- To comment out a swatch of text, use \omitit{blah blah blah} or \eat{ xxx }
\long\def\omitit#1{}
\newcommand\eat[1]{}
\newcommand\bs{\bigskip}

%-- Inline title; useful for sub-sub-sections in which you don't want a separate
%-- line for the title.
\newcommand{\inlinesection}[1]{\smallskip\noindent{\textbf{#1.}}}

%-- todo notes

\newenvironment{outlineenv}{\par\color{black}}{\par}
\newenvironment{pagelenenv}{\par\color{red}}{\par}

%\newcommand{\outline}[1]{\begin{outlineenv}#1\end{outlineenv}}
\newcommand{\paralenblah}[1]{\begin{pagelenenv}Estimated length: #1
paragraphs\end{pagelenenv}}
\newcommand{\pagelenblah}[1]{\begin{pagelenenv}Estimated length: #1
pages\end{pagelenenv}}
\newcommand{\outline}[1]{\textsf{\textbf{\leavevmode\color{Red}[Outline: #1]}}}
\newcommand{\todoA}[1]{\textsf{\textbf{\color{VioletRed}{[#1]}}}}
\newcommand{\todoB}[1]{\textsf{\textbf{\color{Blue}{[#1]}}}}
\newcommand{\todoC}[1]{\textsf{\textbf{\color{Green}{[#1]}}}}
\newcommand{\todoD}[1]{\textsf{\textbf{\color{Red}{[#1]}}}}
\newcommand{\todoE}[1]{\textsf{\textbf{\color{Orange}{[#1]}}}}
%\newcommand{\todoF}[1]{\textsf{\textbf{\color{VioletRed}{[#1]}}}}
\newcommand{\rough}[1]{\textit{\color{Orange}{#1}}}

\newcommand{\paralen}[1]{
  \ifthenelse{\equal{\showcomments}{1}}{
\paralenblah{#1}}{}}

\newcommand{\pagelen}[1]{
  \ifthenelse{\equal{\showcomments}{1}}{
\pagelenblah{#1}}{}}

\newcommand{\outlinetext}[1]{
  \ifthenelse{\equal{\showcomments}{1}}{
\outline{#1}}{}}

\newcommand{\evan}[1]{
  \ifthenelse{\equal{\showcomments}{1}}{
\todoA{Evan: #1}}{}}
\newcommand{\raja}[1]{
  \ifthenelse{\equal{\showcomments}{1}}{
\todoB{Raja: #1}}{}}
\newcommand{\ok}[1]{
  \ifthenelse{\equal{\showcomments}{1}}{
\todoC{Orran: #1}}{}}
\newcommand{\peter}[1]{
  \ifthenelse{\equal{\showcomments}{1}}{
\todoD{Peter: #1}}{}}

\newcommand{\jethro}[1]{
  \ifthenelse{\equal{\showcomments}{1}}{
\todoE{Jethro: #1}}{}}


\newcommand{\todo}[1]{
  \ifthenelse{\equal{\showcomments}{1}}{
\todoA{TODO: #1}}{}}

% eg, ie, etc, et al ....
\newcommand{\eg}{\emph{e.g.}\xspace}
\def\etc{etc.\@\xspace}
\newcommand{\cf}{{cf.}\xspace}
\newcommand{\ie}{\emph{i.e.}\xspace}
\newcommand{\etal}{\emph{et al.}\xspace}

%-- Little numbers in circles...  Maybe possible to define smarter macro, but
%-- simple attempts didn't work.  There are also UTF-8 characters that can be used.
%-- Might need this to get it working:
%\usepackage{pifont}
\newcommand{\circone}{\protect\raisebox{-0.5pt}{\ding{192}}\xspace}
\newcommand{\circtwo}{\protect\raisebox{-0.5pt}{\ding{193}}\xspace}
\newcommand{\circthree}{\protect\raisebox{-0.5pt}{\ding{194}}\xspace}
\newcommand{\circfour}{\protect\raisebox{-0.5pt}{\ding{195}}\xspace}
\newcommand{\circfive}{\protect\raisebox{-0.5pt}{\ding{196}}\xspace}
\newcommand{\circsix}{\protect\raisebox{-0.5pt}{\ding{197}}\xspace}
\newcommand{\circseven}{\protect\raisebox{-0.5pt}{\ding{198}}\xspace}
\newcommand{\circeight}{\protect\raisebox{-0.5pt}{\ding{199}}\xspace}
\newcommand{\circnine}{\protect\raisebox{-0.5pt}{\ding{200}}\xspace}
\newcommand{\circten}{\protect\raisebox{-0.5pt}{\ding{201}}\xspace}

% Little numbers in circles plus background colors
\newcommand{\colorcirc}[2]{\protect\raisebox{-0.5pt}{\protect\tikz{\protect\path
[fill=#1] (0,0.004) circle [radius=0.105];\protect\node [inner sep=0,outer
sep=0] at (0,0) {\small{#2}};}}}




\newcommand{\squeezeup}{\vspace{-6.0mm}}
%%% Local Variables:
%%% mode: plain-tex
%%% TeX-master: "hotnets15_submit_evolvability"
%%% End:
